\documentclass{article}
\usepackage{fontspec}
\usepackage[final]{microtype}
\usepackage{polyglossia}
\usepackage{fullpage}
\usepackage{hyperref}
\usepackage{listings}
\usepackage{color}

\definecolor{mygreen}{RGB}{0, 150, 0}
\definecolor{myblue}{RGB}{0, 0, 200}
\definecolor{myred}{RGB}{150, 0, 0}

\lstset{ %
  basicstyle=\footnotesize\ttfamily,        % the size of the fonts that are used for the code
  breakatwhitespace=false,         % sets if automatic breaks should only happen at whitespace
  breaklines=true,                 % sets automatic line breaking
  commentstyle=\color{mygreen},    % comment style
  frame=single,	                   % adds a frame around the code
  keepspaces=true,                 % keeps spaces in text, useful for keeping indentation of code (possibly needs columns=flexible)
  keywordstyle=\bfseries\color{myblue},       % keyword style
  language=Java,                 % the language of the code
  numbers=left,                    % where to put the line-numbers; possible values are (none, left, right)
  rulecolor=\color{black},         % if not set, the frame-color may be changed on line-breaks within not-black text (e.g. comments (green here))
  stringstyle=\color{myred},     % string literal style
  showspaces=false,
  showstringspaces=false,
  tabsize=4	                   % sets default tabsize to 2 spaces
}

\author{Renato Utsch e Victor Diniz}
\title{Documentação Trabalho Prático 1\\Programação Modular}

\begin{document}
\maketitle

\section{Introdução}
O trabalho prático consistiu na implementação de um sistema de gerenciamento de estacionamento de veículos automotivos.

O programa implementado para resolver esse problema foi dividido em duas partes: a especificação do estacionamento (os preços, a quantidade de vagas, a relação dos tipos de vagas, etc) e o sistema de controle do estacionamento. Para especificar o estacionamento uma \emph{DSL (Domain Specific Language)} simples foi criada. Essa DSL especifica o número de andares, os tipos de vaga, a quantidade de vagas por andar, o preço de cada vaga e as relações entre as vagas. Uma classe lê esses dados do arquivo de especificação e constrói a classe de controle baseada nesse arquivo. O sistema de controle, por sua vez, possui funções para parar e retirar carros do estacionamento.

\section{Implementação}
As seções abaixo explicam separadamente a especificação do estacionamento e a implementação do sistema de controle.

\subsection{Especificação do Estacionamento}
Ao invés de implementar os preços e os tipos de vaga do estacionamento diretamente no código fonte, o que seria difícil de modificar, resolvemos criar um arquivo de configuração que é lido na inicialização do programa que contém todas as informações necessárias para o estacionamento funcionar. Com isso, 
A especificação do estacionamento foi feita em uma DSL criada especificamente para essa tarefa. O arquivo de configuração padrão do trabalho é mostrado abaixo:

\lstinputlisting[caption=parkinglot.config]{../parkinglot.config}

O arquivo é dividido em 3 seções:
\begin{enumerate}
    \item Linha 1, especifica quantos andares têm no estacionamento, no caso, 4.
    \item Linhas 3 a 7, especifica cada um dos tipos de vaga, junto com quantas vagas desse tipo há por andar e quanto que custa cada vaga. Por exemplo, há duas vagas do tipo MT por andar, e parar nelas custa 3,50 a hora.
    \item Linhas 9 a 13, especificam, para cada um dos tipos de veículo, qual a preferência de vagas para parar no estacionamento. Por exemplo, para veículos do tipo MT, a preferência é primeiro para tentar parar em vagas do tipo MT, depois em vagas do tipo VP e, finalmente em vagas do tipo VG.
\end{enumerate}

É interessante notar que, da forma que o arquivo de configuração é estruturado, a preferência é preencher todas as vagas de um tipo, em todos os andares, antes de passar para um outro tipo da relação. No caso, se preenche todas as vagas do tipo MT, em todos os andares, antes de tentar preencher uma vaga do tipo VP com um veículo do tipo MT.

Para ler esse arquivo, criamos uma classe chamada \emph{ParkingLotBuilder}, que, através de regexes, lê o arquivo de entrada e monta uma estrutura de dados com todas as informações necessárias para o funcionamento do estacionamento. Essa estrutura de dados é então passada para o construtor do sistema de controle (a classe \emph{ParkingLot}) para ser utilizada no sistema de estacionamento.

\subsubsection{Estrutura de Dados}

Com os dados do arquivo de configuração, a classe ParkingLotBuilder constrói uma estrutura de dados do tipo abaixo:

\begin{lstlisting}
HashMap<String, ArrayList<ArrayList<ParkingSpot>>>
\end{lstlisting}

Como essa estrutura é um pouco complicada, vamos descrever ela de dentro para fora:

\begin{itemize}
    \item A classe \emph{ParkingSpot} é responsável por cuidar de uma vaga em particular do estacionamento. Quando o ParkingLotBuilder lê o arquivo de configuração, ele cria uma instância da ParkingSpot por vaga de estacionamento. Essa classe sabe qual o seu tipo de vaga, em qual andar está, se está ocupada ou não, há quanto tempo está ocupada e qual o custo por hora para parar nela.
    \item A \emph{ArrayList<ParkingSpot>} é uma array de ParkingSpots que contém todos os parking spots de um mesmo tipo. Por exemplo, pode haver uma ArrayList com ParkingSpots do tipo "MT". Esses ParkingSpots são ordenados em ordem, de modo que se você iterar por eles primeiro você acessará os do primeiro andar, depois do segundo, e assim por diante.
    \item A \emph{ArrayList<ArrayList<ParkingSpot>} é uma array que contém arrays de ParkingSpots. Isso é necessário pois um determinado tipo de veículo pode parar em tipos diferentes de vaga (por exemplo, um veículo do tipo MT pode parar, em ordem, nas vagas do tipo MT, VP ou VG). As arrays estão ordenadas nessa ordem, de modo que encontrar uma vaga para um determinado tipo de veículo significa apenas iterar pelas arrays de ParkingSpots.
    \item O \emph{HashMap} que encapsula a estrutura de dados descrita no item anterior mapeia a String do tipo de veículo para as arrays com as vagas que esse tipo de veículo pode parar.
\end{itemize}

\subsection{Sistema de controle}
O sistema de controle do estacionamento tira vantagem do formato da estrutura de dados gerada pela classe Builder para funcionar de maneira extremamente simples. O código abaixo exemplifica como parar um veículo:

\begin{lstlisting}
void park(Vehicle v, HashMap<String, ArrayList<ArrayList<ParkingSpot>>> hash) throws ParkingException {
    // Primeiro, encontrar a lista de arrays do tipo de veículo sendo parado.
    ArrayList<ArrayList<ParkingSpot>> arrays = hash.get(v.type());

    // Segundo, iterar por essa lista para receber as listas de vagas de um tipo.
    for(ArrayList<ParkingSpot> spots : arrays) {
         // Terceiro, para cada tipo de vaga, iterar por todas as vagas.
         for(ParkingSpot s : spots) {
             // Quarto, se o local não está ocupado, parar lá.
             if(!s.isOccupied()) {
                 s.occupy(v);
                 return;
             }
         }
    }

    throw ParkingException("Estacionamento lotado.");
}
\end{lstlisting}

A partir do exemplo acima, podemos definir o seguinte método para parar um veículo:
\begin{enumerate}
    \item Recuperar a lista de arrays de vagas do tipo do veículo sendo parado. Note que como tipos de veículos diferentes podem compartilhar os mesmos tipos de vagas, listas de arrays de vagas diferentes podem referenciar as mesmas vagas.
    \item Iterar pelas listas de arrays para recuperar as arrays representando apenas um tipo de vaga.
    \item Iterar por cada vaga do mesmo tipo até encontrar uma vaga livre. Note que se as vagas de um tipo estiverem todas preenchidas, basta apenas continuar iterando pelas outras arrays para achar vagas disponíveis de outro tipo.
    \item Se encontrar uma vaga vazia, parar o carro nela e retornar.
    \item Se nenhuma vaga for encontrada, disparar uma exceção.
\end{enumerate}

Essa estrutura de dados é muito versátil por englobar todas as informações necessárias para o estacionamento de uma maneira bem simples de recuperar e iterar, inclusive com ótima complexidade assintótica. Tudo isso é possível porque listas de arrays de vagas podem compartilhar as mesmas arrays de vagas (por exemplo, veículos do tipo MT compartilham as listas de vagas do tipo VG com os tipos de veículo VP e VG).

Para tirar um veículo do estacionamento, o código é igualmente simples, basta iterar por todas as vagas em que o tipo de veículo pode parar até encontrar o veículo com a placa procurada.

\section{Testes}
Para testar os resultados do sistema de gerenciamento de estacionamento, utilizamos da biblioteca JUnit. Essa é uma biblioteca de \emph{unit testing} comumente utilizada pelas aplicações Java, que permite especificar casos de teste e o que é esperado da aplicação dado o caso de teste.

Com a JUnit, pudemos especificar vários casos de teste (na pasta \emph{test}) que verificavam se, dada uma entrada particular para o estacionamento, a saída era a esperada. A partir disso, foi possível fazer diversas modificações no código com confiança de que tudo continuava funcionando, pois os testes nos garantiam que nossas modificações não mudavam o resultado esperado.

\section{Conclusão}
A implementação desse sistema de estacionamento foi uma boa forma de aprender a utilizar a linguagem Java e colocar alguns dos conceitos aprendidos em aula em prática. Embora seja bastante simples, foi o suficiente para aprendermos como trabalhar em equipe e boas práticas de Orientação a Objetos.

\nocite{*}
\bibliographystyle{ieeetr}
\bibliography{doc}

\end{document}

